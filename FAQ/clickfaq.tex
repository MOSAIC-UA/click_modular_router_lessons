\documentclass[a4paper]{article}
\usepackage[english]{babel}
\usepackage{amsmath}
\usepackage{amsthm}
\usepackage{amsfonts}
\usepackage{url}
\usepackage{hyperref}
\usepackage{listings}
\lstset{language=C++,linewidth=\columnwidth,breaklines=true,basicstyle=\footnotesize,tabsize=2,frame=l,framesep=15pt,xleftmargin=20pt}

\author{Johan Bergs -- Jeremy Van den Eynde}
\title{Click Frequently Asked Questions}
\date{}

\begin{document}
\maketitle
\tableofcontents

\section{General} % (fold)
\label{sec:general}

\subsubsection*{Which sources apart from this FAQ and the Click tutorial are good to learn more about Click?} % (fold)
The Click Modular Router homepage \url{http://www.read.cs.ucla.edu/click/}. You can  also search the mailinglist with Google: \texttt{site:pdos.csail.mit.edu whatyousearch}.


\subsubsection*{I have asked everyone and no one can help me with my Click problems. Can
I now ask for help on the Click malinglist?}

NO. Mail to Johan Bergs and Jeremy Van den Eynde, they'll
help you. (If everyone mails simple questions to the mailinglist then
the high quality of it will drop, with less and worse Click releases as
a consequence.)

\subsubsection*{I run Click on my 64bit AMD Opteron and configure fails saying that it
does not know the machine `x86\_64-unknown'?}

Replace `config.sub' by /usr/share/automake/config.sub and rerurn
configure. This version is probably more up to date and will support
your architecture.

\subsubsection*{On my AMD Athlon64 I get a compilation error in aqm/red.cc about a cast
from void* to int.}

You are running Click-1.4.3! It was not yet ready for 64 bit systems,
where this type of casts is not permitted. Replace the line with a
correct cast:
\begin{lstlisting}
switch ((intptr* t)vparam) {
\end{lstlisting}


\subsubsection*{Click does not compile on (Open)Solaris}

Click can be compiled on a Solaris system, with some tweaks to the
makefile. But Fenix is very slow and the platform does not correspond to
the reference platform. We strongly discourage you to compile on Fenix.

% section general (end)


\section{Scripts}

\subsubsection*{ToDump writes to a PCAP dumpfile to analyse things in Wireshark, but nothing
reasonable appears in the dumps?}

The timestamp is not set on your packets. Using the SetTimeStamp element
in front of the ToDump element will solve this.

\subsubsection*{ToDump logs only a fraction of what passes through it?}

Do not log with multiple ToDumps to the same file, as only one of them
will write out passed packets properly. Use separate files for each
ToDump element.

\subsubsection*{I want to dump IP packets that are not encapsulated in Ethernet headers.
How can I do that without encapsulating with fake Ethernet headers
first?}

Take a look at the ENCAP keyword for the ToDump element.

\subsubsection*{I can't use that EtherSwitch or ListenEtherSwitch element, Click does
not find it?}

You should enable the etherswitch package while configuring Click, so:
./configure ---disable-linuxmodule ---enable-local ---enable-etherswitch

\subsubsection*{Is there syntax highlighting for click scripts in my favourite editor?}

Most probably not directly, but syntax highlighting for C++ works well
and makes reading click scripts sometimes a little bit easier.
\emph{Hint:Kwrite/Kate allows you to expand and collapse compound
elements, which can help reading large scripts.}

\subsubsection*{So, you seem to need to run packets through something like CheckIPHeader
in order to fill out the packet header annotations, correct? Do these
annotations persist across elements that use ``uniquify'', things like
Tee, etc.?}

Yes.

\subsubsection*{Are transport-layer annotations only filled out once you run through
something like IPClassifier?}

The CheckIPHeader annotation sets both the start-network-header
annotation and the end-network-header==start-transport-header
annotation.

\subsubsection*{Why am I getting messages that no route is found for 0.0.0.0?}

If packets do not contain an IP destination annotation, the routing
table will try to route those packets based on an empty annotation,
which equals trying to find a route for the address 0.0.0.0. So setting
that annotation is crucial.

\subsubsection*{I want to broadcast packets. Should I add a route for 255.255.255.255?}

No! There are multiple reasons. Broadcast packets should not be routed
at all, as they are link local. You will also know the L2 address, most
probably Ethernet broadcast. Sending these packets to an ARPQuerier is
not a good idea. Also, with routers with multiple interfaces, where are
you going to send the packet to?

\subsubsection*{I get messages that no route is found for 255.255.255.255?}

Think about why you get these messages and take a good look at the
previous question.

\section{API}

\subsubsection*{Where can I find a browsable version of the Click code?}

The doxygen documentation is located at \url{http://www.read.cs.ucla.edu/click/doxygen/}

\subsubsection*{How do I use a hashmap without compilation errors?}

Do not forget to include bighashmap.cc in your sourcefile with the right
arguments. It's best to do it at the bottom, so you get something like:

\begin{lstlisting}
// macro magic to use bighashmap
#include <click/bighashmap.cc>
#if EXPLICIT_TEMPLATE_INSTANCES 
template class HashMap <Packet*, IPAddress*> ;
#endif  
CLICK_ENDDECLS  
EXPORT_ELEMENT(MyElement)
\end{lstlisting}


\subsubsection*{Am I right in thinking that variables like ``network\_length()''
represent the length from the beginning of the network header to the end
of the packet?}

Yes!

\subsubsection*{Am I right in assuming there will never be any layering information
(e.g., ethernet footer) after the data?}

In practice yes, although this depends on your network layer.

\subsubsection*{Does changing the annotation variables (e.g.,
ip\_header()-\textgreater{}ip\_len) actually modify the packet itself?}

Nope!

\subsubsection*{How can I use multiple structs to construct a packet?}

This is actually a matter of doing pointer arithmetics. Yes it's C, yes
it's dangerous but yes you need it.\\ Say you have a packet format for
the first part, defined in the struct FirstPart. The next part of the
packet is defined in the struct NextPart.\\ When you want to create a
packet with those headers encapsulated in Ethernet, IP and UDP, you will
do almost the same as with a packet consisting of one struct. The only
difference if the packet size and of course the way to access the
data.

\begin{lstlisting}
int tailroom = 0;
int packetsize = sizeof(FirstPart)+sizeof(SecondPart);
int headroom = sizeof(click_eth)+sizeof(clickip)+sizeof(clickudp);
WritablePacket* packet = Packet::make(headroom, 0,packetsize, tailroom);
if (packet == 0 )return click_chatter( "cannot make packet!");
memset(packet->data(), 0, packet->length());
FirstPart* format = (FirstPart*) packet->data();
format->something = somevalue;
NextPart* nextformat = (NextPart*) (packet->data()+sizeof(FirstPart));
nextformat->somethingelse = someothervalue; 
\end{lstlisting}

Note that a different way to achieve this is doing:
\begin{lstlisting}
struct NextPart{
SomethingElseType somethingelse;
}
struct FirstPart{
SomethingType something;
// ...
struct NextPart;
}

\end{lstlisting}

This also works but it limits the flexibility, with the first solution
you can easily manage dynamic packet sizes.

\section{New elements}

\subsubsection*{My new element is not found, it does not appear in elements.conf. What's
going on?}

Formatting: check if no space ended up before EXPORT\_ELEMENT. If that's
the case Click ignores that macro so your element is not added.

\subsubsection*{I have made a directory below elements to group a number of elements,
but these elements are not found by Click.}

Look in configure.in for ELEMENTS\_ARG\_ENABLE, these are the statements
for the inclusion of element directories. Now add a line of the form:
\begin{lstlisting}
ELEMENTS\_ARG\_ENABLE(dirname, {[}comments{]}, NO)
\end{lstlisting}
The
last option is either YES or No and determines whether the elements in
your directory should be included by default when executing configure.
\\ When you write YES here: execute configure again, if you do not want
to include elements in the new directory then add the option
---disable-dirname to your configure statement.\\ When you write NO
here: execute configure again, but only when adding ---enable-dirname
these elements will be included. Do not forget to execute \\
\texttt{      make elemlist} again when adding more elements to the
directory.

\subsubsection*{I have moved an element or changed my directory structure and all of a
sudden I get errors that Click can not find the required files for my
element?}

In that case you need to run a make clean and execute configure again,
of course with the right options enabled or disabled. On top in
`config.log' you can find which parameters you used last time, if you do
not remember them anymore.

\section{Compilation and link errors}

\subsubsection*{Click does not want to compile any more, the compilation error is
located in the `beetlemonkey' function?}

Your compiler output will look similar to this:
\begin{lstlisting}
elements.cc: In function `Click::Element* beetlemonkey(unsigned int)':
elements.cc:284: error: ISO C++ forbids declaration of `type name' with no type
elements.cc:284: error: uninitialized const in `new' of `const int'
elements.cc:284: error: cannot convert `const int*' to `Click::Element*' in return
elements.cc:285: error: cannot convert `char**' to `Click::Element*' in return
elements.cc:286: error: syntax error before `(' token
elements.cc:287: error: ISO C++ forbids declaration of `type name' with no type
elements.cc:287: error: uninitialized const in `new' of `const int'
elements.cc:287: error: cannot convert `const int*' to `Click::Element*' in return
elements.cc: In function `void click_export_elements()':
elements.cc:582: error: syntax error before `(' token
elements.cc:586: error: `class_name' undeclared (first use this function)
elements.cc:586: error: (Each undeclared identifier is reported only once for each function it appears in.)
elements.cc:586: error: syntax error before `::' token
elements.cc:588: error: syntax error before `(' token
elements.cc: In function `void click_unexport_elements()':
elements.cc:906: error: syntax error before `(' token
elements.cc:908: error: syntax error before `::' token
elements.cc:909: error: syntax error before `(' token
gmake[1]: *** [elements.o] Error 1 
\end{lstlisting}

In the header file of one of your elements the method `class\_name' is
not defined on 1 line. This will work:\\

\begin{lstlisting}
const char* class_name() const { return "OLSRNeighborInfoBase"; }
\end{lstlisting}
The following will NOT work:\\
\begin{lstlisting}
const char* class_name() const
{
return "OLSRNeighborInfoBase";
}
\end{lstlisting}
To solve this, you have to find out which element contains the errors,
which is possible by analyzing the compiler output to elements.cc. You
will see that something is wrong there. The element that causes the
problems can be located quickly. Make sure `class\_name' is fixed and
rerun \texttt{       make elemlist all} to recompile. After this those
compiler errors should be gone.\\ \emph{Hint:Take care with editors that
format your source code, they might not leave `class\_name' on 1 line.}

\subsubsection*{My error looks like the previous one but I certainly have everything on
1 line?}

Check if no extra spaces appear after the () in EXPORT\_ELEMENT. That
``macro'' is very sensitive.

\subsubsection*{My error looks like the previous one but EXPORT\_ELEMENT is fine too.
Click complains that certain elements do not exist?}

Errors like this one:
\begin{lstlisting}
CXX elements.cc
elements.cc: In function
Click::Element* beetlemonkey(uintptr_t)':
elements.cc:349: error:
Print' has not been declared
make[1]: *** [elements.o] Error 1
\end{lstlisting}
or
\begin{lstlisting}
elements.cc: In function 'Element* beetlemonkey(uintptr_t)':
elements.cc:259: error: expected type-specifier before 'MyNullPush'
elements.cc:259: error: cannot convert 'int*' to 'Element*' in return
elements.cc:259: error: expected ';' before 'MyNullPush'
elements.cc:259: error: 'MyNullPush' was not declared in this scope
make[1]: *** [elements.o] Error 1
\end{lstlisting}

arise when an includeguard is not copied well from the original code.
Most probably the same name is used in 2 elements. So check your
\#ifndef and \#define macros.

\subsubsection*{I get linkerrors about a vtable in one of my elements?}

Errors like this one:
\begin{lstlisting}
../elements/pdhcp/pdhcp_discover_generator.cc:14: undefined reference to
vtable for PDHCPDiscoverGenerator'
../elements/pdhcp/pdhcpdiscovergenerator.cc:14: undefined reference to
vtable for PDHCPDiscoverGenerator'
collect2: ld returned 1 exit status
\end{lstlisting}
are caused by not declaring or defining your element constructor and
destructor.

\subsubsection*{gcc says the type ErrorHandler is not defined?}

\begin{lstlisting}
../elements/mobileiphelpers/dynamicetherencap.cc:47: error: invalid use of undefined type struct `ErrorHandler'
\end{lstlisting}

Add this include at the top of your code:
\begin{lstlisting}
#include <click/error.hh>
\end{lstlisting}

\subsubsection*{I get syntax error inside Click code, to be exact the includes?}

Check whether you included click/config.h before everything else. If you
did not, some macros are undefined with strange syntax errors as a
consequence.

\subsubsection*{I get strange constructor deprecation errors?}

If you get errors like this:
\begin{lstlisting}
	../elements/local/simplepullelement.cc: In constructor 'SimplePullElement::SimplePullElement()':
	../elements/local/simplepullelement.cc:8: warning: '_basector ' is deprecated (declared at ../include/click/element.hh:201)
\end{lstlisting}

then you are using old Click-1.4.3 element syntax. You should use the
Click-1.6 syntax as described in the slides.

\subsubsection*{gcc starts complaining about pcap libraries during linking?}

You are using code that is compiled on a different machine, and the
object files refer to libraries on different locations. A make clean
should solve this.

\section{Endianness - htonl/htonl - htons/ntohs - uintX\_t}

\subsubsection*{Some fields in my packets are completely wrong, they seem reversed?}

A network uses \emph{big endian}-byte order while e.g. the systems in
the pc labs use \emph{little endian} as they run the intel architecture.
You can use the following functions to convert integers in network-byte
order:
\begin{lstlisting}
unsigned short htons(unsigned short hostshort); // 16 bit
unsigned long htonl(unsigned long hostlong); //32-bit
\end{lstlisting}

To convert from network byte order to system byte order you can use:
\begin{lstlisting}
unsigned short ntohs(unsigned short networkshort); // 16 bit
unsigned long ntohl(unsigned long networklong); //32-bit
\end{lstlisting}

More information on this subject is available at \url{http://en.wikipedia.org/wiki/Htons\#Endianness_in_networking}.

\subsubsection*{What about uint\_64?}

Try to avoid using that, if you need to then you should contact Johan and
Jeremy.

\subsubsection*{There is no such thing as an uint4\_t, and I need this for two adjacent
fields?}

Start from two uint8\_t fields and join those with bit-shifting and
binary operators. Say you need the bit sequence for 01100001, then you
could do this:

\begin{lstlisting}
	uint8_t a = 6; // (00000110)
	uint8_t b = 1; // (00000001)
	uint8_t result = (a << 4) & b;
\end{lstlisting}

\subsubsection*{I convert a number with htonl/htons and then it becomes 0?}

Are you sure the datatype of the field you write in is not uint8? With
uint16 you use htons and for uint32 it's htonl. (You should avoid using
uint64 due to possible alignments issues.)

\subsubsection*{Is this related: I can't get my UDP checksum correct, though everything
is set correctly?}

Are you sure you used htons to set the UDP header length? If not the
checksum can never be set correctly and things will always fail, even if
the checksum is calculated with SetUDPChecksum.

\subsubsection*{Can't we avoid this endianness problem with uint8\_t (8bits), uint16\_t
(16bits) etc?}

Those are data types for unsigned integers with x bits. uint8\_t = 1
byte, so endianness does not have any impact here. For all larger
datatypes, so datatypes consisting of multiple bytes, you must take
endianness into account. In the end, this is about how data is stored in
memory, not about how data should be interpreted.

\section{Data structure alignment}
\subsubsection*{sizeof(struct X) returns the wrong size?}
Because of the way data is structured within the memory of a computer, the compiler wants to align this data properly. A CPU operates with word sized chunks, which are 4 byte long on a 32 bit processor and 8 byte long on a 64 bit processor. Therefore, when you define a struct, the compiler will add padding bytes to any data structure that does not align properly with these word sized chunks. Take following code for example:

\begin{lstlisting}
struct X {
uint8_t a;
uint8_t b;
uint16_t c;
};
\end{lstlisting}

On a 32 bit system sizeof (struct X) would return 4. However, on a 64 bit system, this would return 8, as 8 bytes is the smallest multiple of the word length (8) that can represents 4 bytes. If you want to be able to work with the actual size of a struct instead of the padded one, the solution is to define your struct as follows:

\begin{lstlisting}
struct __attribute__((__packed__)) X {
uint8_t a;
uint8_t b;
uint16_t c;
};
\end{lstlisting}

This statement will instruct your compiler to only make the struct as large as it needs to be to hold all data (actually, on a per-byte basis), ignoring alignment. Now, sizeof (struct X) will always return 4.

\section{Segfaults and other runtime errors}

\subsubsection*{My code crashes, how do I start debugging?}

Take a look at valgrind, ddd and gdb. If you need more information about
these tools just post a question or ask Joha and jeremyn.

\subsubsection*{My elements configures without errors and still I get an unspecified
error?}

Most probably you get something like: While configuring `aninstance ::
SomeElement': unspecified error Check if your configure method returns 0
at its end, it has to return an integer you know.

\subsubsection*{When running Click, it crashes on timer initialization code?}

Are you sure you are initializing your timer in the initialize method of
your element (as in the examples)? (Initializing your timers in the
configure method of an element is a bad idea. Then the timer will look
for its router object, which is not completely configured because it is
configuring its elements with their configure method. To avoid this
dependency circle you should initialize your timer in the elements
initialize method and everything will work properly.)

\subsubsection*{My element duplicates a packet (it is sent on multiple output ports or
is used multiple times) and a little bit later I get segmentation
faults?}

If you duplicate packets you have to clone them too. Otherwise the
memory assigned to the packet is freed multiple times, causing
segfaults.

\subsubsection*{My code is written using every possible standard, but still it crashes
when generating the first packet?}

In which method are you generating this packet? You should not generate
any packets in the configure or initialize method. At that moment, the
router is not yet initialized completely and other elements may still
have to be initialized. This means things can and will crash.

\section{Assignment and RFCs}

\subsubsection*{I have seen that the KULeuven/\ldots{} has put a protocol implementation
online. Can I use it? Or base my code on it?}

You searched very well, but that's not the purpose of the project. By
the way, that implemenation is complete, it can do more than the
assignments requires. Making everything compatible with the scripts is a
lot of work\ldots{} We know a lot of implemenations and we are actively
looking for new ones, as with every project copying equals fraud and as
a consequence reduces your points to 0.

\subsubsection*{I changed an address in an IP header and now my UDP checksum is
incorrect?}

The UDP RFC states:

\begin{quote}
Checksum is the 16-bit one's complement of the one's complement sum of a
pseudo header of information from the IP header, the UDP header, and the
data, padded with zero octets at the end (if necessary) to make a
multiple of two octets. The pseudo header conceptually prefixed to the
UDP header contains the source address, the destination address, the
protocol, and the UDP length. This information gives protection against
misrouted datagrams. This checksum procedure is the same as is used in
TCP.
\end{quote}

(So far for the theoretical ISO/OSI layers) So add a SetUDPChecksum
element after changing IP addresses.

\section{Hints for the project}

\subsubsection*{A number of things might be useful to know:}

\begin{itemize}
\item
  ask questions about Click and the RFC as soon as possible
\item
  it's not good if you are stuck too long tell us about problems as soon
  as possible
\item
  also if there are tensions in your project group use the Click code as
  much as possible
\item
  almost never do you have to change existing elements, mail us if you
  think you have to
\item
  for the final evaluation everything has to compile, run without
  crashing and work conform the RFC, if software does not compile or run
  we can't give you points!
\item
  if you prepare the final tar.gz, test it on a different machine before
  testing, one header file is easy to forget
\item
  etherswitch is already included in Click, don't send that element to
  us
\item
  make sure you can work with handlers and write down commonly used
  handlers in a file so you don't have to retype them again and again
\item
  avoid hardcoded constants like 0x22 throughout your code, define it
  once (with a macro or a constant) to make things much more clear
\item
  use Clicks headers for IP, UDP, \ldots{}
\end{itemize}

To highlight this one again: use existing Click code where you can, even
if the task is very small. Nothing is easier than pushing out a packet
and having the small task done by another element. You are sure that it
works and don't need to wonder whether e.g. you first need to set a
checksum to 0 to calculate it. So use existing code to:

\begin{itemize}
\item
  encapsulate a packet (dynamic destination? control the Encap element
  with handlers!)
\item
  update checksums (UDP, IP, \ldots{}) or set them after packet creation
\item
  strip some bytes, at the front or at the back of the packet
\item
  set headers
\item
  modify special headers, then use Click's structs instead of yours
\end{itemize}

Hint: tunneling is possible without almost any new code.

\end{document}