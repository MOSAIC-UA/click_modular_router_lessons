\documentclass{beamer}

\usetheme{PATS}
\usepackage[english]{babel}
\usepackage[utf8]{inputenc}
\usepackage[T1]{fontenc}
\usepackage{cmap}
\usepackage{listings}

\setbeamercovered{transparent}

\title{Click Protocol Implementations}
\subtitle{Implementing Protocols with Click Modular Router}
\author{Johan Bergs \and Jeremy Van den Eynde}
\institute{
		University of Antwerp\\
		imec - IDLab Research Group
}
\date{October 2017}

\logo{\includegraphics[height=0.75cm]{figures/mosaic_logo.png}}


\begin{document}


\begin{frame}[t]
\titlepage
\end{frame}


\section*{Outline}

\frame{
  \frametitle{Outline}
  \tableofcontents[subsectionstyle=hide]
}
\section{Introduction} % (fold)
\label{sec:introduction}
\frame{
\frametitle{Disclaimer}
Everyone has his/her own coding style

This presentation outlines our experiences with coding in Click

Strong suggestions towards our students, may be helpful to others
}

% section introduction (end)

% section osi_layers_model_in_click (end)
\section{Packet Parsing/Generation} % (fold)
\label{sec:}

\frame{
\frametitle{Click Software Design}
Click has two classes that are important for creating new protocols:
\begin{itemize}
	\item Packet: handles allocation of memory and access to the raw data.
	\item Element: an abstract class whose children implement specific, preferably bite-sized, actions with Packets.
\end{itemize}
Protocols are implemented using these basic building blocks configured in the Click script. Your design should take this into account. 

\textbf{Don't start from scratch and try to force your own design into Click.}
}

\frame{
\frametitle{Working with Packet Formats}
Packet formats == structs
\begin{itemize}
	\item structs are a typical C concept, very low level
	\item tempting to improve this by wrapping the packets in objects 
	\item attractive to create packet factories
\end{itemize}
Do not do this, very large overhead:
\begin{itemize}
	\item In terms of memory and computation (allocate objects, create and delete objects)
	\item In terms of code base
\end{itemize}
Use the plain structs
\begin{itemize}
	\item Requires getting used to
	\item Straightforward: packet manipulation is low-level 
\end{itemize}
}

\frame{
\frametitle{What about the special cases?}
If you really need globally available methods for the struct
\begin{itemize}
	\item define procedure in file containing the struct (mind the include guards!)
\end{itemize}
}

% section  (end)
\section{General Click Patterns}
\frame{
\frametitle{The Infobase}
Click architecture is centered around elements, however sometimes information has to be shared between elements

Solutions:
\begin{itemize}
	\item Global variables
	\item Handler calls
	\item Calls on public element methods
	\item \textbf{Create an Infobase element}
\end{itemize}
Infobase:
\begin{itemize}
	\item Multiple Elements need to have access to a shared data source
	\item Infobase Element is only responsible for sharing 
	\item No input or output ports
	\item Passed to Elements who need it using the configuration
\end{itemize}

}
\frame{
\frametitle{The source-responder pattern}
Generating packets can be hard, especially when packet formats must adhere to a standard. Difficult to debug, as packets can get lost in routing complexity.

\begin{itemize}
	\item Debug bit-by-bit, manually
	\item Create a source element, only responsible for packet generation
	\item MySource -> ToDump(my.dump) -> Discard
\end{itemize}

Packet generation logic is now in one place, and can easily be tested. Next create a packet responder that parses the packet and responds. When tested, plug the elements into a larger Click script.

\emph{Other names: generator-receiver, sender-processor}
}

\frame{
\frametitle{Testing generated packets}
Multiple suggested possibilities:
\begin{itemize}
	\item Avoid plain Print
	\item Use ToDump, a good generator will allow for easy dumping!
	\item Take a look at the testie framework (in /test) for Click standard testing
\end{itemize}
Beware of testing overhead!
}
\section{General Click Anti-patterns}
\frame{
\frametitle{God Elements}
A single element which implements an entire protocol

Not a good idea:
\begin{itemize}
	\item Large amount of code in one file
	\item No reuse of existing Click elements
	\item Protocol logic becomes hard to debug
	\item Cooperative development more difficult   
\end{itemize}
}
\frame{
\frametitle{Other anti-patterns}
\begin{itemize}
	\item Reinventing the wheel, especially for checksums
	\item Packet and other factories
	\item Over-engineering
\end{itemize}
Conclusion: remain low level!
}
\end{document}


