\documentclass{beamer}

\usetheme{PATS}
\usepackage[english]{babel}
\usepackage[utf8]{inputenc}
\usepackage[T1]{fontenc}
\usepackage{cmap}
\usepackage{listings}

\setbeamercovered{transparent}

\title{Layered Protocols in Click}
\subtitle{Implementation of Layered Protocols with Click Modular Router}
\author{Johan Bergs \and Jeremy Van den Eynde}
\institute{
		University of Antwerp\\
		imec - IDLab Research Group
}
\date{October 2017}

\logo{\includegraphics[height=0.75cm]{figures/mosaic_logo.png}}


\begin{document}


\begin{frame}[t]
\titlepage
\end{frame}


\section{ISO/OSI Reference Model}

\frame{
  \frametitle{Click in ISO/OSI Reference Model}
	\begin{figure}[t]
		\centering
		\includegraphics[width=8cm]{figures/clickinosimodel.pdf}
	\end{figure}
}

\section{IP Router}

\frame{
\frametitle{IP Router}
\begin{figure}[t]
	\centering
	\includegraphics[width=11cm]{figures/iprouter.pdf}
\end{figure}
}

\frame{
\frametitle{Input Path}
\begin{minipage}[t]{0.5\linewidth}

\begin{itemize}
	\item Handles ARP
	\item Classifies
	\begin{itemize}	
		\item ARP Requests
		\item ARP Replies
		\item Data
	\end{itemize}		
\end{itemize}

\end{minipage}
\begin{minipage}[t]{0.45\linewidth}
\begin{figure}[t]
	\centering
	\includegraphics[width=6cm]{figures/inputpath.pdf}
\end{figure}
\end{minipage}

}

\frame{
\frametitle{Routing Path}
\begin{minipage}[t]{0.5\linewidth}

\begin{itemize}
	\item Remove Ethernet Header
	\item Check if the IP header is correct
	\item Routes the packet
	\begin{itemize}	
		\item local
		\item forwarding path
	\end{itemize}		
\end{itemize}

\end{minipage}
\begin{minipage}[t]{0.45\linewidth}
\begin{figure}[t]
	\centering
	\includegraphics[width=5cm]{figures/routingpath.pdf}
\end{figure}
\end{minipage}
}

\frame{
\frametitle{Output Path}
\begin{minipage}[t]{0.5\linewidth}

\begin{itemize}
	\item Check if the packet should be forwarded
	\item If not send an ICMP error
	\item Routes the packet
	\begin{itemize}	
		\item GW
		\item TTL exceeded
		\item Fragmentation
	\end{itemize}		
	\item else forward the packet
		\item via ARPQuerier
\end{itemize}

\end{minipage}
\begin{minipage}[t]{0.45\linewidth}
\begin{figure}[t]
	\centering
	\includegraphics[width=6cm]{figures/outputpath.pdf}
\end{figure}
\end{minipage}
}

\frame{
\frametitle{IP Router}
\begin{figure}[t]
	\centering
	\includegraphics[width=11cm]{figures/iprouter.pdf}
\end{figure}
}

\section{Respecting the Layered Model}
\frame{
\frametitle{Layered Moded in Click}
\emph{"With great power comes great responsibility."}
\begin{itemize}
	\item Click receives raw ethernet frames from Layer 2 and handles all the processing up until it is handed to the higher layers.
	\item Click can also handle Transport layer headers such as intercepting UDP and TCP packets making it possible to implement daemons inside Click.
	\item The user is in charge of respecting the layered model, not Click.
\end{itemize}
Try to adhere to the ISO/OSI reference layer as much as possible. This makes sure that the packets you handle are what you expect.
}

\frame{
\frametitle{Following the Layered Model}


\begin{minipage}[t]{0.5\linewidth}
	e.g. What if you need to capture UDP packets on port 7. 
	\begin{itemize}
		\item Capturing them after the Routing Path decides the packet has reached its destination.
		\item This ensures that you capture valid IP packets with the correct destination.
	\end{itemize}
\end{minipage}
\begin{minipage}[t]{0.45\linewidth}
\begin{figure}[t]
	\centering
	\includegraphics[width=5cm]{figures/routingpath.pdf}
\end{figure}
\end{minipage}
}

\frame{
\frametitle{Breaking the Layered Model}
\begin{minipage}[t]{0.5\linewidth}
	If you would capture the packet directly from the input (interface):
	\begin{itemize}
		\item You could be processing the wrong packet the wrong way (e.g. treat an ARP message as UDP).
		\item You would have to add all these checks yourself in Click or even worse inside your own element.
	\end{itemize}
\end{minipage}
\begin{minipage}[t]{0.45\linewidth}
\begin{figure}[t]
	\centering
	\includegraphics[width=5cm]{figures/inputpath.pdf}
\end{figure}
\end{minipage}

}

\end{document}


