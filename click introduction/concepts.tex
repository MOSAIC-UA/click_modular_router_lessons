\documentclass{beamer}

\usetheme{PATS}
\usepackage[english]{babel}
\usepackage[utf8]{inputenc}
\usepackage[T1]{fontenc}
\usepackage{cmap}
\usepackage{listings}
\usepackage{verbatim}
\setbeamercovered{transparent}
\usepackage{listings}
\usepackage{hyperref}

\lstset{basicstyle=\footnotesize\ttfamily,breaklines=true}

\title{Click Concepts}
\subtitle{Click Modular Router Concepts and Philosophy}
\author{Johan Bergs \and Jeremy Van den Eynde}
\institute{
		University of Antwerp\\
		imec - IDLab Research Group
}
\date{October 2017}

\logo{\includegraphics[height=0.75cm]{figures/mosaic_logo.png}}


\begin{document}
\lstset{breaklines}

\begin{frame}[t]
\titlepage
\end{frame}


\section*{Outline}

\frame{
  \frametitle{Outline}
  \tableofcontents[subsectionstyle=hide]
}

\section{Background}
\frame{
\frametitle{Network coding}
\begin{itemize}
	\item Inside router: packets being processed while flowing through routers
	\item Classical network software: operate on packets, procedural programming style
	\item Code reuse? Lots of returning concepts
\end{itemize}
}
\frame{
\frametitle{Click Modular Router}
\begin{itemize}
	\item Extensible toolkit for writing packet processors
	\item PhD thesis dr.~Eddie Kohler (MIT)
	\item Architecture centered on elements
	\begin{itemize}
		\item Small building blocks
		\item Perform simple operations e.g. decrease TTL
	\end{itemize}
	\item Click routers
	\begin{itemize}
		\item Directed graphs of elements
	\end{itemize}
\end{itemize}
}
\frame{
\frametitle{Why not a daemon?}
\begin{itemize}
	\item I'll just code it as a daemon.
	\begin{itemize}
		\item How modular is your result?
	\end{itemize}
	\item Designing elements is hard, in the beginning
	\begin{itemize}
		\item Follow the framework, it's there already
	\end{itemize}
	\item Other alternatives:
	\begin{itemize}
		\item System calls are hard
		\item Which libraries for the packet formats?
		\item The Linux kernel API is hard
		\item Let's not talk about kernel development
	\end{itemize}
\end{itemize}
}
\frame{
\frametitle{This has already been done}
\begin{itemize}
	\item So have linked lists
	\begin{itemize}
		\item Did you do real network programming before?
		\item E.g., do you know how ARP really works?
	\end{itemize}
	\item Click is never used in real life
	\begin{itemize}
		\item It works at layer 3, it is in the network
		\item Being used in production at large companies, links with e.g. Google
	\end{itemize}
\end{itemize}
}
\section{Routers and Elements} % (fold)
\label{sec:click_routers}

\frame{
\frametitle{Click Routers}
\begin{itemize}
	\item Router: Elements connected by edges
	\item Output ports to input ports
	\item Describes possible packet flows
	\item \lstinline!FromDevice(eth0)->Counter->Discard;!
\end{itemize}
\begin{figure}[t]
	\centering
	\includegraphics[width=5cm]{figures/graph.pdf}
\end{figure}
}
\frame{
\frametitle{Element Classes}
\begin{itemize}
	\item Class: element type (reuse!)
	\item Configuration string: initializes this instance
	\item Input port(s): Interface where packets arrive, triangles 
	\item Output port(s): Interface where packets leave, squares
	\item Instances can be named: myTee :: Tee
\end{itemize}
\begin{figure}[t]
	\centering
	\includegraphics[width=5cm]{figures/element.pdf}
\end{figure}
}
\frame[allowframebreaks]{
\frametitle{Element ports}
Push port:
\begin{itemize}
	\item Filled square or triangle
	\item Source initiates packet transfer: event based packet flow
\end{itemize}
Pull port:
\begin{itemize}
	\item Empty square or triangle
	\item Destination initiates packet transfer
	\item Used with polling, scheduling, ...
\end{itemize}
Agnostic port:
\begin{itemize}
	\item Square-in-square or triangle-in-triangle
	\item Becomes push or pull
\end{itemize}
\begin{figure}[t]
	\centering
	\includegraphics[width=8cm]{figures/pushpull.pdf}
\end{figure}
}
\frame[allowframebreaks]{
\frametitle{Push-pull violations}
Push port 
\begin{itemize}
	\item has to be connected to push or agnostic port
	\item Conversion from push to pull with push-to-pull element
	\item E.g. queue
\end{itemize}
Pull port
\begin{itemize}
	\item Has to be connected to pull or agnostic port
	\item Conversion from pull to push with pull-to-push element
	\item E.g. unqueue
\end{itemize}
\begin{figure}[t]
	\centering
	\includegraphics[width=8cm]{figures/pushpullviolations.pdf}
\end{figure}

}
\frame{
\frametitle{Compound elements}
\begin{itemize}
	\item Group elements in larger elements
	\item Configuration with variables
	\item Pass configuration to the internal elements, can be anything (constant, integer, elements, IP address, ...)
	\item Motivates reuse
\end{itemize}
\begin{figure}[t]
	\centering
	\includegraphics[width=5cm]{figures/compound.png}
\end{figure}
}
% section click_routers (end)
\section{Implementation Philosophy} % (fold)
\label{sec:implementation}
\frame{
\frametitle{Element classes - Routers}
Element classes?
\begin{itemize}
	\item Elements (actually element classes): C++ classes
	\item Element instantations: C++ objects
\end{itemize}
Click Routers:
\begin{itemize}
	\item Click router configurations (or short Click routers): text files
	\item parsed when starting Click, Click builds object graph of elements
\end{itemize}
}
\begin{frame}[fragile]
\frametitle{Click Graphs}
Text files describing the Click graph:
\begin{itemize}
	\item Elements with their configurations
	\item Compound elements
	\item Connections between elements
\end{itemize}
\begin{lstlisting}
    src :: FromDevice(eth0); ctr :: Counter; 
    sink :: Discard; 
    src -> ctr; ctr -> sink;
\end{lstlisting}
or
\begin{lstlisting}
    FromDevice(eth0) -> Counter -> Discard;
\end{lstlisting}
\end{frame}
% 
% 

\begin{frame}[fragile]
\frametitle{Input and output ports}
Identified by number (0,1,..)
\begin{itemize}
	\item Input port: \verb!-> [nr1]Element ->!
	\item Output port: \verb!-> Element[nr2] ->!
	\item  Both: \verb!-> [nr1]Element[nr2] ->!
	\item  Only one port: number can be omitted
\end{itemize}
Motivates instance naming
\begin{lstlisting}
mypackets::IPClassifier(dst host $myaddr,-);
FromDevice(eth0) -> mypackets;
mypackets[0]-> Print(mine)->[0]Discard;
mypackets[1]-> Print("the others") -> Discard;
\end{lstlisting}
\end{frame}

\begin{frame}[fragile]
\frametitle{Compound elements in Click scripts}
\begin{lstlisting}
elementclass DumbRouter { $myaddr |
    mypackets :: IPClassifier(dst host $myaddr,-);
    input[0]-> mypackets; mypackets[0]-> [1]output;
    smypackets[1]-> [0]output;
}
u :: DumbRouter(1.2.3.4);
FromDevice(eth0)-> u;
u[0]-> Discard;
u[1]-> ToDevice(eth0);
\end{lstlisting}
\end{frame}

\frame{
\frametitle{Element Configuration}
Listed in click script
\begin{itemize}
	\item First required arguments
	\item Then optional arguments
	\item Then arguments by keyword (after keyword)
\end{itemize}
Lots of data types supported
\begin{itemize}
	\item Integers 
	\item Strings e.g. "data"        
	\item IP addresses 143.129.77.30
    \item Elements
\end{itemize}
}
\frame{
\frametitle{Element Configuration Examples}
\begin{itemize}
	\item SimpleElement("data")
	\item SimpleElement("data", ACTIVE false)
	\item SimpleElement("moredata", 800)
    \item SimpleElement("data", 800, DATASIZE 67, SOURCE 1.2.3.4)
\end{itemize}
}
\frame{
\frametitle{Packets}
Packet consists of payload and annotations, payload:
\begin{itemize}
	\item raw bytes (char*)
	\item  Access with struct*
\end{itemize}
Annotations: metadata to simplify processing, ``post-its''
\begin{itemize}
	\item  E.g. start of IP header or TCP header
	\item  Paint annotations
	\item  User defined annotations
\end{itemize}
\begin{figure}[t]
	\centering
	\includegraphics[width=5cm]{figures/annotations.pdf}
\end{figure}
}
% section implementation (end)
\section{Handlers} % (fold)
\label{sec:handlers}
\frame{
\frametitle{Handlers}
Like function calls to an element
\begin{itemize}
	\item ReadHandler: request a value from an element
	\item WriteHandler: pass a string to an element
	\item (There is no ReadWriteHandler: you can't call a ReadHandler with arguments)
\end{itemize}
Can be called from other elements or through socket
\begin{itemize}
	\item Take a look at Pokehandlers!
\end{itemize}
}

\section{Running Click} % (fold)
\label{sec:running_click}
\frame{
\frametitle{Use the source}
\begin{itemize}
	\item Source code: Open source
	\item Runs on Linux, Mac OS X, BSD and partially in Windows
	\item Makefile based
	\item Very little external dependencies (in-kernel)
\end{itemize}
}
\frame{
Kernel module
\begin{itemize}
	\item Completely overrides Linux routing
	\item High speed, requires root permissions
	\item Crashing Click = crashing kernel = crashing system
\end{itemize}
Userlevel
\begin{itemize}
	\item Runs as a daemon on a Linux system
	\item Easy to install and still fast
	\item Recommended for this course
\end{itemize}
nsclick
\begin{itemize}
	\item Runs as a routing agent within the ns-2 network simulator
	\item Multiple routers on 1 system
	\item Difficult to install but less hardware needed
\end{itemize}
}
% \frame{
% Send packets to and receive packets from the network
% Connects to the network interface
% Delivered to Click at FromDevice(ethX)
% Sent to system by ToDevice(ethX)
% Send packets to and receive packets from the system 
% With a virtual tun/tap interface
% Set the default route to the tun/tap device to make sure traffic passes through the Click route
% Warning: Linux routing is still working
% Addresses (IP and MAC) must be correct
% Make sure IPs from Click are different from system Ips
% 
% }
% section running_click (end)

\section{References} % (fold)
\label{sec:references}
\frame{
\frametitle{References}
\begin{itemize}
	\item Click website: \url{http://www.read.cs.ucla.edu/click/}
	\item Element documentation (by name or category)
	\item Programming Concepts
	\item Doxygen documentation ('Internals documentation')
\end{itemize}
}
\frame{
\frametitle{Click PhD thesis}

PhD thesis: comprehensive documentation of every concept

Interesting chapters: 
\begin{itemize}
	\item Introduction
	\item Architecture: elements, packets, connections, push and pull, element implementation
	\item Language: syntax, configuration strings, compound elements
\end{itemize}

Click mailinglist: \url{http://librelist.com/browser/click/} (Large source of information, mainly for Click developers)
}
% section references (end)
\end{document}


